\documentclass{article}
\usepackage{amsmath}
\usepackage{amssymb}
\usepackage{graphicx}
\usepackage{listings}
\usepackage{hyperref}

\title{Advanced Examples and Use Cases}
\order{7}

\begin{document}


\section{Advanced Examples and Use Cases}

The \texttt{hyperon-miner} repository also contains code corresponding to a few richer pattern mining scenarios, illustrating how you can adapt and extend the basic mining pipeline for different domains and custom requirements.

\subsection{Mining Attribute Co-occurrence in the Sample Dataset}

The file \texttt{data/sample.metta} defines a toy population of ''humans'' with attributes such as \texttt{man}/\texttt{woman}, \texttt{ugly}/\texttt{beautiful}, and \texttt{sodadrinker}.  You can mine this dataset for surprising attribute co-occurrences:

\begin{verbatim}
;; 1. Start a fresh AtomSpace and load the sample data
!(import! &db hyperon-miner:experiments:data:small-ugly)

;; 2. Mine up to 2-clause patterns, requiring at least 5 matches
!(frequency-pattern-miner &kb &dbspace &specspace &cndpspace &aptrnspace &conjspace 2 0)

\end{verbatim}

Expected output might include:

\begin{verbatim}
(,(inherit $X sodadrinker) (inherit $X man))    ; freq: 12, I-surprisingness: +1.2
(, (inherit $X sodadrinker) (inherit $X ugly))   ; freq:  8, I-surprisingness: +0.8
\end{verbatim}

\subsection{Specialized Pipeline via frequent-pattern-miner.metta}

The script \texttt{experiments/rules/frequent-pattern-miner.metta} automates data loading, mining frequent patterns:

\begin{verbatim}
! (import! &dbspace   hyperon-miner:experiments:data:ugly_man_sodaDrinker)
! (import! &self hyperon-miner:experiments:rules:build-specialization)
! (import! &self hyperon-miner:experiments:rules:candidate-patterns)
! (import! &self hyperon-miner:experiments:rules:conjunction-expansion)
! (frequent-pattern-miner &kb &dbspace &specspace &cndpspace &aptrnspace &conjspace 10 1)
\end{verbatim}

Running this yields a list of the most frequent patterns in the dataset.

\subsection{Conjunction Expansion}

To experiment with pattern-joining/conjunction strategies, call the expand_conjunction function in the conjunction module directly:

\begin{verbatim}
! (import! &self hyperon-miner:experiments:rules:candidate-patterns)
! (import! &self hyperon-miner:experiments:rules:conjunction-expansion)
! (import! &conj-exp) #python module 
! (= (minsup) 5)
! (= (depth) Z)
;; 1. Generate 1-clause candidates above support threshold 5
!(candidatePatterns &dbspace &specspace (minsup) &cndpspace)
;; 2. Expand to 2-clause patterns with custom rules
!(expand_conjunction $conjunct $pattern &dbspace (minsup) 2 False) 
;;conjuct and #pattern are candidate patterns
;; alternatively if we want to perform conjunction expansion on all the candidate patterns for however many clauses we want all at once
!(do-conjunct &dbspace $cndpspace $conjunct (minsup) (depth)) 
;;conjuct is our starting pattern and we conjoin it with all the patterns in &cndpspace recursively
\end{verbatim}

This lets you expand two patterns into conjunction by creating a common variable between them and control which variable merges are allowed before support filtering.

\subsection{Type-Constrained Mining via Dependent Types}

The \texttt{dependent-types/} folder shows how to enforce structural constraints on patterns:

\begin{verbatim}
;; 1. Load the dependent-type miner module
!(import! &kb hyperon-miner:dependent-types:MinerCurriedDTL)

;; 2. Instantiate a type-aware miner
;; Initialize miner with the given db, kb and parameters
!(init-miner &db &kb (ms) (highsurp))
;; 3. Mine patterns respecting DTL rules
;; frequent patterns 
!(miner &db (ms) (depth) (highsurp))

;; Surprising patterns
!(miner-surprising &db (ms) (depth) (highsurp))

;;This may not run here ,explore it in the hyperon-miner:dependent-types folder.
\end{verbatim}

Only patterns meeting your dependent-type invariants will survive.


\subsection{Empirical Truth-Value (EMPTV) Scoring}

When your AtomSpace uses STV atoms, convert pattern probabilities into STV judgments:

\begin{verbatim}
;; 1. Import the EMPTV components 
! (import! &self experiments:utils:emp-tv-bs)
! (import! &self experiments:utils:bs-utils)
! (import! &self experiments:utils:common-utils)
! (import! &self experiments:utils:beta-dist)
! (import! &self experiments:utils:constants)
! (import! &self experiments:utils:TruthValue)
! (import! &self experiments:utils:surp-utils)
! (import! &self experiments:utils:miner-utils)
! (import! &self experiments:utils:gen_partition)
! (import! &self experiments:rules:est-tv)
! (import! &self experiments:rules:emp-tv)
! (import! &db experiments:data:ugly_man_sodaDrinker)

;;Pattern: 
(= (pattern) (, (Inheritance $x $z) (Inheritance $x human)))
!(emp-tv  (pattern) &db )
\end{verbatim}

\subsection{Truth-Value Estimation (ESTV) }

Component to calculate the probability (or truth value) estimate of a pattern

\begin{verbatim}
;;Import the est-tv components, but comment out the already loaded ones.
;;Since everything is already loaded above, no need to import anything here.
!(do-ji-tv-est  &db (pattern) $emp-tv)

;;this computes the joint-independent truth value estimate of the pattern
\end{verbatim}

You will get simple truth value of the pattern :

\begin{verbatim}
(STV probability confidence)
\end{verbatim}

\subsection{JSD-Based Surprisingness}

You can use a Jensen-Shannon Divergence measure from \texttt{experiments/rules/jsd-surpr}:

JSD helps you to calculate the difference between the empirical truth value and the truth value estimate of a pattern. Therefore, we can take this difference as the surprisingness of the pattern.

\begin{verbatim}
! (import! &self experiments:utils:util-jsd)
! (import! &self experiments:rules:jsd)

!(do-jsd $emp-tv $est-tv) 
\end{verbatim}

This surfaces clause pairs whose joint distribution diverges most from independence.

\end{document}